\nonstopmode{}
\documentclass[a4paper]{book}
\usepackage[times,inconsolata,hyper]{Rd}
\usepackage{makeidx}
\usepackage[utf8]{inputenc} % @SET ENCODING@
% \usepackage{graphicx} % @USE GRAPHICX@
\makeindex{}
\begin{document}
\chapter*{}
\begin{center}
{\textbf{\huge Package `wflsa'}}
\par\bigskip{\large \today}
\end{center}
\inputencoding{utf8}
\ifthenelse{\boolean{Rd@use@hyper}}{\hypersetup{pdftitle = {wflsa: Weighted Fused LASSO Signal Approximator (wFLSA)}}}{}
\begin{description}
\raggedright{}
\item[Title]\AsIs{Weighted Fused LASSO Signal Approximator (wFLSA)}
\item[Version]\AsIs{1.2}
\item[Date]\AsIs{2024-10-01}
\item[Author]\AsIs{Louis Dijkstra [aut, cre]}
\item[Maintainer]\AsIs{Louis Dijkstra }\email{dijkstra@leibniz-bips.de}\AsIs{}
\item[Description]\AsIs{A package for computing the 
Weighted Fused LASSO Signal Approximator (wFLSA)}
\item[Imports]\AsIs{}
\item[Depends]\AsIs{R (>= 4.0.2), Rcpp}
\item[LinkingTo]\AsIs{Rcpp}
\item[License]\AsIs{GPL-3}
\item[Encoding]\AsIs{UTF-8}
\item[LazyData]\AsIs{true}
\item[RoxygenNote]\AsIs{7.2.3}
\item[URL]\AsIs{}\url{https://github.com/bips-hb/wflsa}\AsIs{}
\item[BugReports]\AsIs{}\url{https://github.com/bips-hb/wflsa/issues}\AsIs{}
\end{description}
\Rdcontents{\R{} topics documented:}
\inputencoding{utf8}
\HeaderA{calculate\_diagonal\_matrix\_A}{Determine the Diagonal of Matrix \eqn{A}{}}{calculate.Rul.diagonal.Rul.matrix.Rul.A}
%
\begin{Description}\relax
The wFLSA algorithm requires the choice of a matrix \eqn{A}{} such that 
\eqn{A - D'D}{} is positive semidefinite.  
We choose the matrix \eqn{A}{} to be diagonal with a fixed value \eqn{a}{}. 
This function determines the smallest value of \eqn{a}{} such that 
\eqn{A - D'D}{} is indeed positive semidefinite. We do this be determining 
the largest eigenvalue
\end{Description}
%
\begin{Usage}
\begin{verbatim}
calculate_diagonal_matrix_A(W, eta1, eta2)
\end{verbatim}
\end{Usage}
%
\begin{Arguments}
\begin{ldescription}
\item[\code{W}] Weight matrix \eqn{W}{}

\item[\code{eta1, eta2}] The values \eqn{\eta_1 = \lambda_1 / \rho}{} and 
\eqn{\eta_2 = \lambda_2 / \rho}{}
\end{ldescription}
\end{Arguments}
%
\begin{Value}
Value of \eqn{a}{}
\end{Value}
%
\begin{References}\relax
Zhu, Y. (2017). An Augmented ADMM Algorithm With Application to the 
Generalized Lasso Problem. Journal of Computational and Graphical Statistics, 
26(1), 195–204. https://doi.org/10.1080/10618600.2015.1114491
\end{References}
\inputencoding{utf8}
\HeaderA{get\_estimate\_lambda1}{Get Estimates for Beta Coefficients for a Given Lambda1 Value}{get.Rul.estimate.Rul.lambda1}
%
\begin{Description}\relax
This function takes the resulting fit from the wflsa function and returns the
estimates for the beta coefficients for all lambda2 values that were considered
before.
\end{Description}
%
\begin{Usage}
\begin{verbatim}
get_estimate_lambda1(fit, lambda1)
\end{verbatim}
\end{Usage}
%
\begin{Arguments}
\begin{ldescription}
\item[\code{fit}] The result object obtained from wflsa function

\item[\code{lambda1}] The lambda1 value for which beta coefficients are needed
\end{ldescription}
\end{Arguments}
%
\begin{Value}
A list containing:
\begin{ldescription}
\item[\code{betas}] Matrix of beta coefficients for each lambda2 value. Each row corresponds to a single lambda2 value
\item[\code{lambda2}] Vector of lambda2 values considered in the fit.
\end{ldescription}
\end{Value}
%
\begin{Examples}
\begin{ExampleCode}
# Example usage:
# fit <- wflsa(data, lambda1_values, lambda2_values)
# estimate <- get_estimate_lambda1(fit, lambda1_value)
# Now you can access estimated beta coefficients for different lambda2 values
\end{ExampleCode}
\end{Examples}
\inputencoding{utf8}
\HeaderA{get\_unique\_values}{Get unique absolute values of a vector after rounding}{get.Rul.unique.Rul.values}
%
\begin{Description}\relax
This function takes a numeric vector, calculates its absolute values,
rounds them to a specified number of digits, finds the unique values,
sorts them, and returns the sorted unique absolute values.
\end{Description}
%
\begin{Usage}
\begin{verbatim}
get_unique_values(vector, digits = 10)
\end{verbatim}
\end{Usage}
%
\begin{Arguments}
\begin{ldescription}
\item[\code{vector}] A numeric vector.

\item[\code{digits}] Number of digits to round to (default is 10).
\end{ldescription}
\end{Arguments}
%
\begin{Value}
A numeric vector containing the sorted unique absolute values.
\end{Value}
%
\begin{Examples}
\begin{ExampleCode}
vector <- c(1.23456789, -1.23456788, 1.23456787, -1.23456786, 1.23456785, -1.23456784)
sorted_unique_abs_vals <- get_sorted_unique_abs_values(vector)
print(sorted_unique_abs_vals)
\end{ExampleCode}
\end{Examples}
\inputencoding{utf8}
\HeaderA{print.wflsa.fit}{Print Function for the Weighted Fused LASSO Signal Approximator (wFLSA) fit object}{print.wflsa.fit}
%
\begin{Description}\relax
This function prints information about the fitted wFLSA object, including the 
number of variables, the number of lambda pairs, and the estimated beta coefficients.
\end{Description}
%
\begin{Usage}
\begin{verbatim}
## S3 method for class 'wflsa.fit'
print(fit, ...)
\end{verbatim}
\end{Usage}
%
\begin{Arguments}
\begin{ldescription}
\item[\code{fit}] An object of class wflsa.fit.

\item[\code{...}] Additional arguments to be passed to the print function.
\end{ldescription}
\end{Arguments}
\inputencoding{utf8}
\HeaderA{soft\_threshold}{Apply soft thresholding to a numeric vector}{soft.Rul.threshold}
%
\begin{Description}\relax
This function applies soft thresholding to each element of a numeric vector
using a specified threshold value (\code{lambda1}). Soft thresholding shrinks
values toward zero by applying a penalty when the absolute value of an element
exceeds the threshold.
\end{Description}
%
\begin{Usage}
\begin{verbatim}
soft_threshold(x, lambda1)
\end{verbatim}
\end{Usage}
%
\begin{Arguments}
\begin{ldescription}
\item[\code{x}] A numeric vector.

\item[\code{lambda1}] Threshold value for soft thresholding.
\end{ldescription}
\end{Arguments}
%
\begin{Value}
A numeric vector with each element softened using the specified threshold.
\end{Value}
%
\begin{Examples}
\begin{ExampleCode}
x <- c(3, -2, 5, -4, 1)
lambda1 <- 2
soft_thresholded <- soft_threshold(x, lambda1)
print(soft_thresholded)
\end{ExampleCode}
\end{Examples}
\inputencoding{utf8}
\HeaderA{wflsa}{The Weighted Fused Lasso Signal Approximator (wFLSA) Algorithm}{wflsa}
%
\begin{Description}\relax
Solves the weighted Fused LASSO Signal Approximator optimization problem 
using an ADMM-based approach. The problem is formulated as follows: 
\deqn{
\hat{\beta} = \argmin \frac{1}{2} || y - \beta ||_2^2 + \lambda_1 ||\beta ||_1 + \lambda_2 \sum_{i < j} w_{ij} | \beta_i - \beta_j |  
}{}
where:
\begin{itemize}

\item{} \eqn{y}{} is the response with mean \eqn{0}{}.
\item{} \eqn{\beta}{} is the vector of coefficients to be estimated.
\item{} \eqn{|| \cdot ||_1}{} and \eqn{|| \cdot ||_2}{} are the \eqn{L_1}{}- and \eqn{L_2}{}-norms, respectively.
\item{} \eqn{\lambda_1 > 0}{} is the regularization parameter controlling the strength of the sparsity penalty. 
\item{} \eqn{\lambda_2 > 0}{} is the regularization parameter controlling the smoothness. 
\item{} \eqn{w_{ij} \in [0,1]}{} is the weight between the \eqn{i}{}-th and \eqn{j}{}-th coefficient.

\end{itemize}

This implementation solves the entire path for a fixed value of \eqn{\lambda_2}{}.

An Estimator for the Weighted Fused Lasso Signal Approximator (wFLSA)
\end{Description}
%
\begin{Usage}
\begin{verbatim}
wflsa(
  y,
  W,
  lambda2 = c(0.1, 0.2),
  rho = 1,
  max_iter = 1e+05,
  eps = 1e-10,
  truncate = 1e-04,
  offset = TRUE
)
\end{verbatim}
\end{Usage}
%
\begin{Arguments}
\begin{ldescription}
\item[\code{y}] Vector of length \eqn{p}{} representing the response variable (assumed to be centered).

\item[\code{W}] Weight matrix of dimensions \eqn{p \times p}{}.

\item[\code{lambda2}] Vector of positive regularization parameters for smoothness penalty (Default: \code{(.1, .2)})

\item[\code{rho}] ADMM's parameter (Default: \code{1}).

\item[\code{max\_iter}] Maximum number of iterations (Default: \code{1e5}).

\item[\code{eps}] Stopping criterion. If differences are smaller than \code{eps}, 
the algorithm halts (Default: \code{1e-10}).

\item[\code{truncate}] Values below \code{truncate} are set to \code{0} (Default: \code{1e-4}).

\item[\code{offset}] Logical indicating whether the data is centered (Default: \code{TRUE}).
\end{ldescription}
\end{Arguments}
%
\begin{Value}
A list containing:
\begin{itemize}

\item{} \code{betas}: A list with estimates. Each entry are the \eqn{\beta}{}-estimates
for a specific value of \eqn{\lambda_2}{}. 
\item{} all input variables.

\end{itemize}

\end{Value}
%
\begin{Note}\relax
\strong{Important Note:} The algorithm assumes \eqn{y}{} to be centered, i.e., its mean is 0.
\end{Note}
%
\begin{Author}\relax
Louis Dijkstra
\end{Author}
%
\begin{SeeAlso}\relax
\code{\LinkA{genlassoRcpp}{genlassoRcpp}}
\end{SeeAlso}
%
\begin{Examples}
\begin{ExampleCode}
# Example usage of the wflsa function
y <- c(1, 2, 3)
W <- matrix(c(1, 0, 0, 0, 1, 0, 0, 0, 1), ncol = 3)
lambda2 <- c(0.1, 0.2)
result <- wflsa::wflsa(y, W, lambda2)
wflsa::get_estimate_lambda1(result, lambda1 = .5)
\end{ExampleCode}
\end{Examples}
\printindex{}
\end{document}
